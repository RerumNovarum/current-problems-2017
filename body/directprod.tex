TODO.

В будущем теорему можно обобщить имея в виду, например,
следующий известный результат~\cite{bellman-matrices-kron}.
Пусть матрицы \( X{=}(x_{ij})\in\mathbb{C}^{n\times n},\ Y\in\mathbb{C}^{m\times m} \)
имеют собственные значения
\( \lambda_1, \ldots, \lambda_n \)
и \( \mu_1, \ldots, \mu_m \) соответственно.
Тогда их прямое произведение (произведение Кронекера)
\[
    X\otimes Y =
    \begin{pmatrix}
        x_{11} Y & \cdots & x_{1n} Y \\
        \vdots & \ddots & \vdots \\
        x_{n1} Y & \cdots & x_{nn} Y \\
    \end{pmatrix}\in\mathbb{C}^{mn\times mn}
    \]
имеет спектр
\[
    \sigma(X\otimes Y) = \{ \lambda_i \mu _j;\ i=\overline{1,n},\ j=\overline{1,m} \}.
    \]
