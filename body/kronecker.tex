Теперь рассмотрим произведения Кронекера
\[
    A\otimes B =
    \begin{pmatrix}
        a_{11} B & \cdots & a_{1N} B \\
        \vdots   & \ddots & \vdots \\
        a_{N1} B & \cdots & a_{NN} B
    \end{pmatrix}
    \in \mathbb{K}^{{MN}{\times}{MN}}
\]
    квадратных матриц
\( A={(a_{ij})}\in\mathbb{K}^{N{\times}N},
 \ B={(b_{ij})}\in\mathbb{K}^{M{\times}M}. \)
Мы будем исследовать их спектральное поведение
    под действием малых (в смысле нормы) возмущений:
\begin{equation}\label{nkjpcs-kronperturb}
    A\otimes B - F.
\end{equation}

Заметим, что произведение Кронекера обладает рядом замечательных свойств~\cite{bellman-matrices-kron}:
\begin{itemize}
\item Ассоциативность:
    \[ A\otimes (B\otimes C) = (A\otimes B)\otimes C. \]
\item Дистрибутивность относительно сложения:
    \[ (A+B)\otimes(C+D) = A\otimes C + A\otimes D + B\otimes C + B\otimes D. \]
\item Произведение Кронекера в определённом смысле полилинейно:
    \[ (AB)\otimes(CD) = (A\otimes C)(B\otimes D) \]
    всегда когда произведения \( AB \) и \( CD \) определены.
\item След матрицы \( A\otimes B \) есть \[ \operatorname{tr}(A\otimes B) = \operatorname{tr}A\operatorname{tr}B. \]
\item Если \( A \) и \( B \) --- симметричные,
      то и \( A\otimes B \) симметрична.
\end{itemize}
Заметим также, что ``tiled'' матрица из предыдущего раздела
    может также быть представлена разложена в произведение Кронекера:
\[
    \mathbb{A} =
    \begin{pmatrix}
    A & \cdots & A\\
    \vdots & \ddots & \vdots \\
    A & \cdots & A\end{pmatrix} =
        J_N\otimes A.
    \]

\begingroup
\textbf{Лемма}.
\itshape
Пусть \( A \) и \( B \) --- простой структуры,
    То есть \( A \) имеет \( N \) собственных векторов
    \( f_1, \ldots, f_N \)
    и им соответствуют собственные значения \( \mu_1, \ldots, \mu_N \),
    а \( B \) обладает собственными векторами \( h_1, \ldots, h_M \)
    с собственными значениями \( \lambda_1, \ldots, \lambda_M \).
Тогда матрица \( A\otimes B \) также простой структуры;
    она имеет \( MN \) линейно-независимых собственных векторов
    \( f_i\otimes h_j,\ i{=}\overline{1,N}, j{=}\overline{1,M} \)
    и им соответствуют собственные значения \( \mu_i \lambda_j \).
\endgroup


Пусть среди попарных произведений \( \mu_i \lambda_j \)
    лишь \( s \) чисел различны: \( \nu_1, \ldots, \nu_s \).
Каждому собственному значению \( \nu_k \) (\( k{=}\overline{1,s} \)) соответствует
    инвариантное подпространство
    \[ E_k = \operatorname{span}(f_i\otimes h_j;\ \mu_i\lambda_j = v_k,\ i{=}\overline{1,N},\ j{=}\overline{1,M}) \subset \mathbb{K}^{MN}. \]
Эти подпространства образуют разложение \( \mathbb{K}^{MN} \):
    в прямую сумму:
    \[ \mathbb{K}^{MN} = E_1 \oplus \cdots \oplus E_s. \]
Это означает, что каждый вектор \( x\in\mathbb{K}^{MN} \) может единственным образом
    быть представлен в виде суммы
    \begin{equation}\label{nkjpcs-decomposition-x}
        x = x_1 + \cdots + x_s,\ x_k\in E_k,\ k=\overline{1,s}.
    \end{equation}
Это разложение \( \mathbb{K}^{MN} \)
    соответствует разложению единицы \( E\in \mathbb{K}^{MN{\times}MN} \)
    в сумму матриц спектральных проекторов:
    \[
        E = P_1 + \cdots + P_s.
    \]
Спектральный проектор \( \mathcal{P}_k \) (\(k{=}\overline{1,s}\)) определяется формулой
    \[
        \mathcal{P}_k x = x_k \in E_k\subset \mathbb{K}^{MN}
    \]
    в соответствии с разложением~\eqref{nkjpcs-decomposition-x} вектора \( x \).

Для всякой матрицы \( X\in \mathbb{K}^{MN{\times}MN} \)
    верно следующее тривиальное равенство:
    \[
        X = \sum_{i,j=1}^s P_i X P_j.
    \]

Теперь матрица \( A\otimes B \) может быть представлена в виде:
    \[
        \mathcal{A} = \sum v_j P_j.
    \]

Теперь мы можем воспроизвести рассужденрия общей схемы и получить оценки.

% \begin{center}
% \textbf{Lemma.}
% {\it
Естественно положить
    \[
        JX = \sum_{j=1}^s P_j X P_j.
    \]
Система уравнений
    \[\left\{\begin{aligned}
        & \mathcal{A}\Gamma X - (\Gamma X) \mathcal{A} = X - JX, \\
        & J\Gamma X = 0,\ X\in \mathbb{K}^{MN{\times}MN}
    \end{aligned}\right.\]
    имеет единственное решение
    \[
        \Gamma X = \sum_{1\leq i{\neq}j \leq s} \frac{1}{\nu_i-\nu_j} P_i X P_j.
    \]
    Норма трансформатора \( \Gamma \) легко считается:
    \[
        \|\Gamma\|_{\mathrm{op}} = \gamma = \frac{1}{\min_{1\leq i{\neq}j\leq s}\lvert\nu_i - \nu_j\rvert}
    \]
% \/}
% \end{center}


\begingroup
\textbf{Теорема}.
\itshape
    Рассмотрим возмущённую матрицу~\eqref{nkjpcs-kronperturb}
        \[
            A{\otimes}B - F.
        \]
    Пусть \( A\in\mathbb{K}^{N{\times}N} \) и \( B\in\mathbb{K}^{M{\times}M} \)
        --- простой структуры.
    Пусть \( f_1, \ldots, f_N \) --- собственные векторы \( A \),
        соответствующие собственным числам \( \mu_1, \ldots, \mu_N \)
        и пусть \( h_1, \ldots, h_M \) --- собственные векторы \( B \)
        соответствующие собственным числам \( \lambda_1, \ldots, \lambda_M \).
    Спектр их кронекерова произведения \( A{\otimes}B \)
        состоит из всевозможных попарных произведений \( \mu_i \lambda_j \),
        которым соответствуют собственные векторы \( f_i\otimes h_j \).
    Пусть из этих \( MN \) собственных значений лишь \( s \) различны:
        \( \nu_1, \ldots \nu_s \).

    Предположим
    \[
        \|F\| \leq \frac14 \gamma^{-1} = \frac14 \min_{1\leq i{\neq}j\leq s}\lvert\nu_i - \nu_j\rvert.
    \]

    Тогда \( A{\otimes}B - F \) подобна
    \[ \sum_{k=1}^s \nu_k P_k - JX^o = \sum_{k=1}^s (\nu_k P_k - P_k X^o P_k) \]
    для некоторой \( X^o \in \mathbb{K}^{MN{\times}MN} \),
    \( \|X^o - F\|\leq 3\|F\| \).

    Собственные значения \( A{\otimes}B - F \) содержатся в кругах
    \[
        \Omega_k = \left\{
            \lambda\in\mathbb{C};
            \ \lvert\lambda - \nu_k\rvert \leq 4\|F\|
            \right\},
        \ k{=}\overline{1,s}.
    \]
    В каждом круге находится хотя бы одно собственное значение.

    Пусть \( \nu_k=\mu_{i_k}\lambda_{j_k} \) --- собственное значение \( A{\otimes}B \) кратности \( 1 \),
        то есть ему соответствует единственный собственный вектор \( v_k = f_{i_k}{\otimes}h_{j_k} \).
    Эквивалентно: пусть \( \mu_{i_k} \) --- собственное значение кратности \( 1 \) для матрицы \( A \)
    и \( \lambda_{j_k} \) --- кратности \( 1\) для \( B \).
    Тогда \( A{\otimes}B - F \) имеет собственное значение в круге \( \Omega_k \)
        и ему соответствует собственный вектор \( \hat{v}_k \), лежащий в границах:
    \[
        \|\hat{v}_k - v_k\| \leq 4\gamma \|F\|.
    \]
    Если \( \nu_k \) достаточно изолирован от других точек спектра \( A{\otimes}B \):
    \[
        \min_{l\neq k}
        \lvert
        \nu_k - \nu_l
        \rvert
        \geq 4\|F\|,
    \]
    то \( \nu_k \) --- единственное собственное значение \( A{\otimes}B - F \),
    лежащее в этом круге.
\endgroup

Для примера, вернёмся к ``tiled'' матрице:
\[
    J_N{\otimes}B =
    \begin{pmatrix}
        B & \cdots & B \\
        \vdots & \ddots & \vdots \\
        B & \cdots & B
    \end{pmatrix}.
\]
В этом случае
    \( \nu_1=N \),
    \( \nu_2=0 \).
Пусть \( \lambda_1,\ldots,\lambda_M \)
    --- спектр \( B \).
Спектр \( J_N{\otimes}B \), в соответствии с последней теоремой, есть
    \[
        \sigma(J_N{\otimes}B) = \left\{ \mu_i\lambda_j;\ i{=}\overline{1,2},\ j{=}\overline{1,M}\right\} = \{0\}\cup N\sigma(B).
    \]
Все эти собственные значения, кроме \( 0 \),
    имеют кратность \( 1 \)
    и изолированны друг от друга при больших \( N \).
При этом \( \gamma=\frac1N \).
Отсюда непосредственно следует теорема предыдущего раздела.

Эти результаты могут быть уточнены с помощью теоремы о расщеплении оператора~\cite{baskakov1987theorem},
    который позволяет рассматоривать каждое собственное значение в отдельности.
% The higher the gap between a picked eigenvalue
%     and the rest of the spectrum
%     the higher the precision of the estimate
%     for the corresponding eigenvalue of the perturbed matrix.
    
