Прямое произведение (произведение Кронекера)
    двух матриц
    \( X{=}(x_{ij}){\in}\mathbb{K}^{m_1{\times}n_1},\ \)
    \( Y{\in}\mathbb{K}^{m_2{\times}n_2} \).
    \( X\otimes Y \)
    определено как блочная матрица
\[
    X\otimes Y =
    \begin{pmatrix}
        x_{11} Y & \cdots & x_{1n_1} Y \\
        \vdots   & \ddots & \vdots \\
        x_{m_11} Y & \cdots & x_{m_1n_1} Y \\
    \end{pmatrix}\in\mathbb{K}^{m_1m_1\times n_1n_2}.
    \]

Пусть \( A{=}(a_{ij}){\in}\mathbb{K}^{n\times n} \)
и \( B\in\mathbb{K}^{m\times m} \).
Будем исследовать спектральные свойства
    матрицы
    \[
        A\otimes B = 
        \begin{pmatrix}
            a_{11} B & \cdots & a_{1n} B \\
            \vdots   & \ddots & \vdots \\
            a_{n1} B & \cdots & a_{nn} B \\
        \end{pmatrix}\in\mathbb{K}^{mn\times mn}
        \]
    и возмущённой матрицы
    \begin{equation}\label{ksv:disturbkron}
        A\otimes B - \mathbb{F}.
    \end{equation}

\begin{ksvlem}
    Пусть \( h\in\mathbb{K}^n \) --- собственный вектор матрицы \( A \),
    отвечающий собственному значению \( \lambda \),
    а \( f\in\mathbb{K}^m \) --- собственный вектор,
    соответствующий собственному значению \( \mu \):
    \[
        A h = \lambda h,
        \]
    \[
        B f = \mu f.
        \]
    Тогда \( h\otimes f \in \mathbb{K}^{mn} \)
    --- собственный вектор матрицы \( A\otimes B \),
    отвечающий собственному значению \( \lambda\mu \):
    \[
        (A\otimes B)(h\otimes f) = \lambda\mu\ h\otimes f.
        \]
\end{ksvlem}
\begin{proof}
    Для доказательства достаточно расписать выражение по координатам.
\end{proof}

Заметим, что рассмотренная в предыдущем разделе
    блочная матрица
\(
    \begin{pmatrix}
        B      & \cdots & B \\
        \vdots & \ddots & \vdots \\
        B      & \cdots & B
    \end{pmatrix}
        \)
    представляет собой частный случай \eqref{ksv:disturbkron}:
\[
    \begin{pmatrix}
        B      & \cdots & B \\
        \vdots & \ddots & \vdots \\
        B      & \cdots & B
    \end{pmatrix} =
    J_n \otimes B,
        \]
    где \( J_n \) --- матрица единиц,
    исследованная в первом примере.
