Матрицей смежностей ориентированного графа на \( N \) вершинах
    называется матрица \( A=(a_{ij}), \)
    размера \( N\times N \),
    в которой \( a_{ij} \) --- количество рёбер
    из \( i \)-й вершины в \( j \)-ую.
Марковский процесс случайного блуждания по графу
    приводит к мере центральности
    узлов сети~\cite{bonacich1972factoring}
    (доминирующий собственный вектор матрицы смежностей,
     являющийся также стационарным распределением
     рассматриваемого процесса,
     описывает долю времени, проводимую агентом в каждом узле).
Изначально использовавшийся в \texttt{Google}
    алгоритм \texttt{PageRank}~\cite{ilprints422}
    вычисления этой меры опирается на степенной метод,
    поэтому его скорость сходимости
    определяется отношением абсолютных величин
    первых двух (упорядоченных по убыванию)
    собственных значений матрицы перехода.
Устойчивость стационарного распределения
    определяется \emph{спектральным зазором}
    --- разностью между модулями первых двух
    собственных значений матрицы перехода\marginpar{TODO: cite}.
В~\cite{schwartz2006fluctuation} описан способ
    аппроксимации почти-инвариантных множеств,
    также опирающийся на спектральное разложение.
В~\cite{chakrabarti2008epidemic,wang2003epidemic}
    для модели Susceptible-Infective-Susceptible
    распространения вируса в~сети
    показано, что эффективная скорость
    распространения вируса есть произведение
    спектрального радиуса матрицы смежностей
    и отношения интенсивностей передачи инфекции и исцеления,
    и что в зависимости от положения этой величины относительно единицы
    случится либо эпидемия, либо оздоровление почти всей сети.

Подробно спектральная теория графов и~е\"е приложения
    изложены в~монографии~\cite{cvetkovic1980spectra}.
