Пусть теперь \( A\in \mathbb{K}^{m\times m} \)
Будем рассматривать матрицу
\[
    \mathbb{A} =
    \begin{pmatrix}
        A      & \cdots & A \\
        \vdots & \ddots & \vdots \\
        A      & \cdots & A
    \end{pmatrix}
    \in \mathbb{K}^{mn{\times}mn},
    \]
    составленную из \( n\times n \) блоков
    и возмущённую матрицу
\[
    \mathbb{A} - \mathbb{B}.
    \]

\begin{ksvlem}
    Пусть \( A \) --- обратимая самосопряжённая матрица.
    В этом случае она имеет \( m \) ортонормальных
    собственных векторов \( h_1, \ldots, h_m \)
    (\( \|h_k\|_2=1,\ k{=}\overline{1,m} \)),
    которым отвечают собственные значения\( \lambda_1, \ldots, \lambda_m \).
    \( \lambda_1, \ldots, \lambda_m \)
    --- её собственные значения,
    с соответствующими им собственными векторами
    \( h_1, \ldots, h_m \),

    Тогда спектр матрицы \( \mathbb{A} \) есть
    \[
        \sigma(\mathbb{A}) = \{0\} \cup n\sigma(A)
        = \{ 0 \} \cup \{ n\lambda;\ \lambda\in\sigma(A) \}.
        \]
    Ненулевым собственным значениям \( n\lambda_1, \ldots, n\lambda_m \)
    матрицы \( \mathbb{A} \)
    соответствуют блочные собственные векторы
    \[
        f_j = \frac{1}{\sqrt{n}}(h_k, \ldots, h_k)\in\mathbb{K}^{mn},
        \ j{=}\overline{1,m},
       \]
    а ортобазис ядра образуют ортогональные образу
    векторы
    \[
        f_{j,k} =
        \frac{1}{\sqrt{k(k+1)}}
        (\underbrace{e_j,\ldots,e_j}_{k\ \text{раз}}, -ke_j, 0, \ldots, 0)\in\mathbb{K}^{mn},
        \ k{=}\overline{1,n-1},
        \ j{=}\overline{1,m}.
        \]
    Другими словами, матрица
    \[
        \mathbb{A} = 
        \begin{pmatrix}
            A      & \cdots & A \\
            \vdots & \ddots & \vdots \\
            A      & \cdots & A
        \end{pmatrix}
        \in\mathbb{K}^{mn{\times}mn}
        \]
    подобна блочно-диагональной матрице
    \[
        \mathcal{A} =
        \left(\begin{array}{c|c}
            \operatorname{diag}(n\lambda_1,\ldots,n\lambda_m) & 0 \\ \hline
            0 & \mathbf{0}
        \end{array}\right),
        \]
    с ортогональной матрицей преобразования подобия
    \[
        U = \operatorname{columns}
        \left(f_1, \ldots, f_m, f_{1,1}, \ldots, f_{1,n{-1}}, \ldots, f_{m,n{-}1}\right).
        \]
\end{ksvlem}

Перейдём к изучению возмущённой матрицы.
Матрицы будем рассматривать в блочном виде:
\[
    X =
        \left(\begin{array}{c|c}
            \begin{matrix}
                x_{11} & \cdots & x_{1m} \\
                \vdots & \ddots & \vdots \\
                x_{m1} & \cdots & x_{mm}
            \end{matrix} &
            \begin{matrix}
                x_{1,m+1} \\
                \vdots \\
                x_{m,m+1}
            \end{matrix} \\ \hline
            \begin{matrix}
                x_{m+1,1} &
                \cdots &
                x_{m+1,m}
            \end{matrix} &
            X_{m+1,m+1}
        \end{array}\right),
    \]
где
\( X_{ij}      {=} x_{ij}      {\in} \mathbb{K} \),
\( X_{m{+}1,j} {=} x_{m{+}1,j} {\in} \mathbb{K} \),
\( X_{i,m{+}1} {=} x_{i,m{+}1} {\in} \mathbb{K} \)
для \( 1\leq i,j \leq m \),
а в нижнем правом углу расположен блок
\( X_{m{+}1,m{+}1} {\in} \mathbb{K}^{m(n{-}1){\times}m(n-1)} \).
Как и прежде, в качестве \( J \)
возьмём оператор блочной диагонализации:
\[
    X =
        \left(\begin{array}{c|c}
            \begin{matrix}
                x_{11} &  & 0 \\
                 & \ddots &  \\
                0 &  & x_{mm}
            \end{matrix} &
            \begin{matrix}
                0 \\
                \vdots \\
                0
            \end{matrix} \\ \hline
            \begin{matrix}
                0 & \cdots & 0
            \end{matrix} &
            X_{m+1,m+1}
        \end{array}\right),
    \]
\begin{ksvlem}
    Пусть все числа \( \lambda_1, \ldots, \lambda_m \) различны.
    Тогда тройка \( (\mathbb{K}^{m{\times}n}, J, \Gamma) \)
    с трансформатором \( \Gamma \), действующим по формуле
    \[
        \Gamma X =
        \left(\begin{array}{c|c}
            \begin{matrix}
                0               & \gamma_{12}(X) & \cdots & \gamma_{1m}(X) \\
                \gamma_{21}(X)  & 0              & \cdots & \gamma_{2m}(X) \\
                \vdots          & \vdots         & \ddots & \vdots & \ \\
                \gamma_{m1}(X)  & \gamma_{m2}(X) & \cdots & 0
            \end{matrix} &
            \begin{matrix}
                \gamma_{1,m+1}(X) \\
                \gamma_{2,m+1}(X) \\
                \vdots \\
                \gamma_{m,m+1}(X)
            \end{matrix} \\ \hline
            \begin{matrix}
                \gamma_{m{+}1,1}(X) &
                \gamma_{m{+}1,2}(X) &
                \cdots &
                \gamma_{m{+}1,m}(X)
            \end{matrix} &
            \mathbf{0}
        \end{array}\right),
        \]
    \[
        \gamma_{ij}(X) =
        \left\{\begin{aligned}
            & \frac1N \frac{1}{\lambda_i - \lambda_j} x_{ij},
              \ 1\leq i{\neq}j \leq m{+}1, \\
            & 0,
              \ 1\leq i{=}j \leq m{+}1,
        \end{aligned}\right.
        \]
    \[
        \lambda_{m+1} \coloneqq 0,
        \]
    является допустимой.
    При этом
    \[
        \|\Gamma\|_{\mathrm{op}} =
        \frac1N
        \frac{1}{\min_{1\leq i{\neq}j \leq m{+}1}|\lambda_i - \lambda_j|} =
        \]
    \[
        = \frac1N
         \max\left\{
         \frac{1}{
             \min\limits_{1\leq i{\neq}j \leq m }{|\lambda_i - \lambda_j|}},
         \frac{1}{
             \min\limits_{1\leq j \leq m}{|\lambda_j|}}
         \right\}
        \]
\end{ksvlem}

\begin{ksvthm}
    Пусть
    \[
        \left\| \mathbb{B} \right\|_{\mathrm{op}}
        \leq 
        \frac{N}{4}
         \min\left\{
             \min\limits_{1\leq i{\neq}j \leq m }{|\lambda_i - \lambda_j|},
             \min\limits_{1\leq j \leq m}{|\lambda_j|}
         \right\}.
        \]
    Тогда исследуемая возмущённая матрица
    \[
        \mathbb{A} - \mathbb{B}
        \]
    имеет спектр
    \[
        \sigma\left(\mathbb{A}\right) =
        \left\{
            n\lambda_1 - x_{11}^o, \ldots, n\lambda_m - x_{mm}^o, 0
        \right\}
        \]
    и собственные векторы
    \[
        \hat{f}_j = f_j + \xi_j,
        ,\ j{=}\overline{1,m}
        \]
    \[
        \hat{f}_{j,k} = f_{j,k} + \xi_{j,k},
        \ j{=}\overline{1,m},
        \ k{=}\overline{1,N{-}1},
        \]
    где
    \[
        x_{jj}{\in}\mathbb{C},
        \ \xi_j, \xi_{j,k}{\in}\mathbb{K}^{mn},
        \]
    \[
        |x_{jj}|,\ \|\xi_j\|_2, \|\xi_{j,k}\|_2 \leq
        4 \|\mathbb{B}\|_{\mathrm{op}} 
          \frac1N
          \max\left\{
          \frac{1}{
              \min\limits_{1\leq i{\neq}j \leq m }{|\lambda_i - \lambda_j|}},
          \frac{1}{
              \min\limits_{1\leq j \leq m}{|\lambda_j|}}
          \right\},
        \]
    \[
        j{=}\overline{1,m},
        \ k{=}\overline{1,N{-}1}.
        \]
\end{ksvthm}
