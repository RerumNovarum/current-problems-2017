Приведём в упрощённом виде, учитывающем конечномерность задачи,
    необходимые далее нотацию и утверждения
    метода подобных операторов~\cite{baskakov-harmonic,baskakov1983},
    развиваемого А.~Г. Баскаковым.


Пространство \( \mathbb{K}^n \) 
    будем считать евклидовым с скалярным произведением
    \[
        (x, y){=}\sum_k x_k\overline{y_k},
        \ x{=}(x_1,\ldots, x_n),
        \ y=(y_1,\ldots, y_n)
        \in \mathbb{K}^n
        \]
    и нормой
    \(
        \|x\|_2^2{=}(x,x).
        \)
Символом \( \mathbb{K}^{m\times n} \)
    будем обозначать линейное пространство
    матриц размера \( m{\times}n \)
    с элементами в поле \( \mathbb{K}\in\{\mathbb{R},\mathbb{C}\} \).
Если \( V \) --- банахово пространство с нормой \( \|\cdot\| \),
    то символом \( L(V) \)
    будем обозначать банахову алгебру
    ограниченных линейных операторов \( V\to V \)
    с операторной нормой
\[
    \|A\|_{\mathrm{op}} = \sup_{\substack{\|x\|=1,\\ x\in V}} \|A x\|,\ A\in L(V).
    \]
Наряду с \( L(\mathbb{K}^n) \)
    будем рассматривать изоморфную ему
    банахову алгебру \( \mathbb{K}^{n\times n} \),
    с одной из следующих норм:
    \( \|A\|_{\mathrm{op}} = \sup_{\substack{\|x\|=1,\\ x\in V}} \|A x\|,\ \)
    \( \|A\|_{\mathrm{F}} = \sqrt{\sum_{i,j} |a_{ij}|^2},\ \)
    для матриц \( A{=}(a_{ij})\in\mathbb{K}^{n\times n} \).
Элементы пространства
    \( L(L(V)) \)
    линейных преобразований операторов на \( V \),
    и изоморфного ему пространства \( L(\mathbb{K}^{n\times n}) \),
    будем называть \emph{трансформаторами}
    (терминология М.~Г.~Крейна).

Спектр (множество собственных значений)
    матрицы \( A\in\mathbb{K}^{n\times n} \)
    будем обозначать \( \sigma(A) \).

Матрицы \( A_1, A_2 \in \mathbb{K}^{n\times n} \)
    называют \emph{подобными}, если существует
    обратимая матрица \( U\in\mathbb{K}^{n\times n} \),
    такая что \( A_1 U = U A_2 \).
Подобные матрицы обладают рядом совпадающих спектральных свойств:
    они имеют одинаковый спектр (\( \sigma(A_1)=\sigma(A_2) \)),
    а собственные векторы \( A_2 \) переходят
    в собственные векторы \( A_1 \) под действием
    преобразования подобия \( U \)
    (т.е. \( A_2 x = \lambda x \implies A_1 U x = \lambda U x \)).

Кортеж \( (\mathbb{K}^{n{\times}n}, J, \Gamma) \),
    будем называть \emph{допустимой тройкой}
    для матрицы \( A\in\mathbb{K}^{n\times n} \),
    если выполняются следующие условия:
\begin{enumerate}
\item
    \( J,\Gamma\in L(\mathbb{K}^{n{\times}n}) \),
    причём \( J \) --- проектор.
\item
    \( \Gamma \) на каждой матрице \( X\in \mathbb{K}^{n\times n} \)
    определяется уравнениями
    \[ A\Gamma X - (\Gamma X) A = X - JX, \]
    \[ J\Gamma X = 0. \]
\end{enumerate}

\begingroup
\textbf{Теорема.}\itshape{О подобии (в~общей постановке см.~\cite{baskakov-harmonic,baskakov1983})}
    Рассмотрим матрицы \( A-B \),
        где \( A,B\in \mathbb{K}^{n{\times}n} \).
    Пусть \( (\mathbb{K}^{n\times n}, J, \Gamma) \)
        --- допустимая тройка для матрицы \( A\in\mathbb{K}^{n{\times}n} \)
        и пусть
    \[
        \|B\|_{\mathrm{op}} \|\Gamma\|_{\mathrm{op}} \leq \frac14.
        \]
    Тогда матрица \( A-B \)
        подобна некоторой матрице \( A - J X^o, \)
        где \( X^o\in\mathbb{K}^{n{\times}n} \)
        --- решение нелинейного уравнения
    \[
        X = B\Gamma X - (\Gamma X)J(B + B\Gamma X) + B \equiv \Phi(X),
        \]
        которое можно найти методом простых итераций,
        как предел сходящейся последовательности
        \( \left(X^{(k)} = \Phi^{k}(0) = \Phi(\Phi(\cdots\Phi(0)\cdots));\  k\in\mathbb{N} \right) \).
    Преобразование подобия матрицы \( A-B \)
        в матрицу \( A-JX^o \) осуществляется
        матрицей \( E+\Gamma X^o \),
        где \( E\in\mathbb{K}^{n{\times}n} \) --- тождественная матрица.
    Это решение удовлетворяет следующим ограничениям:
    \[ \|X^o - B\|_{\mathrm{op}} \leq 3\|B\|_{\mathrm{op}}, \]
    \[ \operatorname{spr}(X^o) \leq \|X^o\|_{\mathrm{op}} \leq 4\|B\|_{\mathrm{op}}. \]
\endgroup
