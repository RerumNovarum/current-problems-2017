Символом \( \mathbb{K}^{m\times n} \)
    будем обозначать линейное пространство
    матриц размера \( m\times n \)
    с элементами в поле \( \mathbb{K}\in\{\mathbb{R},\mathbb{C}\} \).
Если \( V \) --- банахово пространство с нормой \( \|\cdot\| \),
    то символом \( L(V) \)
    будем обозначать банахову алгебру
    ограниченных линейных операторов \( V\to V \)
    с операторной нормой
\[
    \|A\|_{\mathrm{op}} = \sup_{\substack{\|x\|=1,\\ x\in V}} \|A x\|,\ A\in L(V).
    \]


Методом подобных операторов (см.~\cite{baskakov-harmonic,baskakov1983}),
 позволяющим для возмущений ``идеального'' объекта, спектральные свойства которого известны,
 найти элемент рассматриваемой алгебры, изоспектральный возмущ\"енному,
 но имеющий более удобную для вычислений структуру...
TODO
