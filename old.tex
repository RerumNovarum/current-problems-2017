\documentclass[12pt]{article}

\usepackage[cp1251]{inputenc} % следующие две строки используются для
\usepackage[english,russian]{babel} % руссификации AmSLaTeX

\textheight=240mm \textwidth=170mm
% The line above messes with low-level tex
% and causes vertical and horizontal page overflows.
% The `geometry` package seems to fix it.
\usepackage[textheight=240mm,textwidth=170mm]{geometry}

% ----
% User-specific packages

\usepackage{mathtools}
\usepackage{amssymb}
\usepackage{amsmath}
% \usepackage{marginnote}
\usepackage{amsthm}
\newtheorem*{ksvthm*}{Теорема}
\newtheorem{ksvthm}{Теорема}
\newtheorem{ksvlem}[ksvthm]{Лемма}
\newtheorem{ksvcrl}[ksvthm]{Следствие}
\theoremstyle{definition}
\newtheorem{ksvdfn}{Определение}

% ----

\begin{document}
\noindent {УДК 517.984.3 : 519.177}

\begin{center}
    \textbf{Исследование спектральных свойств матриц
смежностей орграфов специального вида
методом подобных операторов}\\[3mm]
    \textbf{Сергей Козлуков}\\[2mm]
    \emph{Воронежский Государственный Университет}\\[2mm]
\end{center}

\subsection*{Аннотация}
Для исследования спектральных свойств матриц специального вида,
в том числе матриц смежностей некоторых графов,
используется Метод Подобных Операторов~\cite{baskakov-harmonic,baskakov1983},
развиваемый А.~Г.~Баскаковым.

\subsection*{Спектральные свойства матриц смежностей}
Матрицей смежностей ориентированного графа на \( N \) вершинах
    называется матрица \( A=(a_{ij}), \)
    размера \( N\times N \),
    в которой \( a_{ij} \) --- количество рёбер
    из \( i \)-й вершины в \( j \)-ую.
Марковский процесс случайного блуждания по графу
    приводит к мере центральности
    узлов сети~\cite{bonacich1972factoring}
    (доминирующий собственный вектор матрицы смежностей,
     являющийся также стационарным распределением
     рассматриваемого процесса,
     описывает долю времени, проводимую агентом в каждом узле).
Изначально использовавшийся в \texttt{Google}
    алгоритм \texttt{PageRank}~\cite{ilprints422}
    вычисления этой меры опирается на степенной метод,
    поэтому его скорость сходимости
    определяется отношением абсолютных величин
    первых двух (упорядоченных по убыванию)
    собственных значений матрицы перехода.
Устойчивость стационарного распределения
    определяется \emph{спектральным зазором}
    --- разностью между модулями первых двух
    собственных значений матрицы перехода~\cite{meyer1994sensitivity}.
В~\cite{schwartz2006fluctuation} описан способ
    аппроксимации почти-инвариантных множеств,
    также опирающийся на спектральное разложение.
В~\cite{chakrabarti2008epidemic,wang2003epidemic}
    для модели Susceptible-Infective-Susceptible
    распространения вируса в~сети
    показано, что эффективная скорость
    распространения вируса есть произведение
    спектрального радиуса матрицы смежностей
    и отношения интенсивностей передачи инфекции и исцеления,
    и что в зависимости от положения этой величины относительно единицы
    случится либо эпидемия, либо оздоровление почти всей сети.

Подробно спектральная теория графов и~е\"е приложения
    изложены в~монографии~\cite{cvetkovic1980spectra}.
\subsection*{Метод подобных операторов}
Приведём в упрощённом виде, учитывающем конечномерность задачи,
    необходимые далее нотацию и утверждения
    метода подобных операторов~\cite{baskakov-harmonic,baskakov1983},
    развиваемого А.~Г. Баскаковым.


Пространство \( \mathbb{K}^n \)
    будем считать евклидовым с скалярным произведением
    \[
        (x, y){=}\sum_k x_k\overline{y_k},
        \ x{=}(x_1,\ldots, x_n),
        \ y=(y_1,\ldots, y_n)
        \in \mathbb{K}^n
        \]
    и нормой
    \(
        \|x\|_2^2{=}(x,x).
        \)
Символом \( \mathbb{K}^{m\times n} \)
    будем обозначать линейное пространство
    матриц размера \( m{\times}n \)
    с элементами в поле \( \mathbb{K}\in\{\mathbb{R},\mathbb{C}\} \).
Если \( V \) --- банахово пространство с нормой \( \|\cdot\| \),
    то символом \( L(V) \)
    будем обозначать банахову алгебру
    ограниченных линейных операторов \( V\to V \)
    с операторной нормой
\[
    \|A\|_{\mathrm{op}} = \sup_{\substack{\|x\|=1,\\ x\in V}} \|A x\|,\ A\in L(V).
    \]
Наряду с \( L(\mathbb{K}^n) \)
    будем рассматривать изоморфную ему
    банахову алгебру \( \mathbb{K}^{n\times n} \),
    с одной из следующих норм:
    \( \|A\|_{\mathrm{op}} = \sup_{\substack{\|x\|=1,\\ x\in V}} \|A x\|,\ \)
    \( \|A\|_{\mathrm{F}} = \sqrt{\sum_{i,j} |a_{ij}|^2},\ \)
    для матриц \( A{=}(a_{ij})\in\mathbb{K}^{n\times n} \).
Элементы пространства
    \( L(L(V)) \)
    линейных преобразований операторов на \( V \),
    и изоморфного ему пространства \( L(\mathbb{K}^{n\times n}) \),
    будем называть \emph{трансформаторами}
    (терминология М.~Г.~Крейна).

Спектр (множество собственных значений)
    матрицы \( A\in\mathbb{K}^{n\times n} \)
    будем обозначать \( \sigma(A) \).

Матрицы \( A_1, A_2 \in \mathbb{K}^{n\times n} \)
    называют \emph{подобными}, если существует
    обратимая матрица \( U\in\mathbb{K}^{n\times n} \),
    такая что \( A_1 U = U A_2 \).
Подобные матрицы обладают рядом совпадающих спектральных свойств:
    они имеют одинаковый спектр (\( \sigma(A_1)=\sigma(A_2) \)),
    а собственные векторы \( A_2 \) переходят
    в собственные векторы \( A_1 \) под действием
    преобразования подобия \( U \)
    (т.е. \( A_2 x = \lambda x \implies A_1 U x = \lambda U x \)).

Кортеж \( (\mathbb{K}^{n{\times}n}, J, \Gamma) \),
    будем называть \emph{допустимой тройкой}
    для матрицы \( A\in\mathbb{K}^{n\times n} \),
    если выполняются следующие условия:
\begin{enumerate}
\item
    \( J,\Gamma\in L(\mathbb{K}^{n{\times}n}) \),
    причём \( J \) --- проектор.
\item
    \( \Gamma \) на каждой матрице \( X\in \mathbb{K}^{n\times n} \)
    определяется уравнениями
    \[ A\Gamma X - (\Gamma X) A = X - JX, \]
    \[ J\Gamma X = 0. \]
\end{enumerate}

\begin{ksvthm*}{О подобии (в~общей постановке см.~\cite{baskakov-harmonic,baskakov1983})}
    Рассмотрим матрицы \( A-B \),
        где \( A,B\in \mathbb{K}^{n{\times}n} \).
    Пусть \( (\mathbb{K}^{n\times n}, J, \Gamma) \)
        --- допустимая тройка для матрицы \( A\in\mathbb{K}^{n{\times}n} \)
        и пусть
    \[
        \|B\|_{\mathrm{op}} \|\Gamma\|_{\mathrm{op}} \leq \frac14.
        \]
    Тогда матрица \( A-B \)
        подобна некоторой матрице \( A-X^o, \)
        где \( X^o\in\mathbb{K}^{n{\times}n} \)
        --- решение нелинейного уравнения
    \[
        X = B\Gamma X - (\Gamma X)J(B + B\Gamma X) + B \equiv \Phi(X),
        \]
        которое можно найти методом простых итераций,
        как предел сходящейся последовательности
        \( X^{(k)} = \Phi^{k}(X) = \Phi(\Phi(\cdots\Phi(X)\cdots)) \).
    Преобразование подобие матрицы \( A-B \)
        в матрицу \( A-JX^o \) осуществляется
        матрицей \( E+\Gamma X^o \),
        где \( E\in\mathbb{K}^{n{\times}n} \) --- тождественная матрица.
    Это решение удовлетворяет следующим ограничениям:
    \[ \|X^o - B\|_{\mathrm{op}} \leq 3\|B\|_{\mathrm{op}}, \]
    \[ \operatorname{spr}(X^o) \leq \|X^o\|_{\mathrm{op}} \leq 4\|B\|_{\mathrm{op}}. \]
\end{ksvthm*}
\subsection*{Пример 1. Матрица смежностей, почти-полного орграфа}
Рассмотрим граф, заданный матрицей смежностей
\[ A = \mathcal{J}_N - B =
   \begin{pmatrix}1 & \cdots & 1\\ \vdots & \ddots & \vdots \\ 1 & \cdots & 1\end{pmatrix} - B,
   \]
    где \( \mathcal{J}_N \)~--- матрица, составленная из \( N{\times}N \) единиц,
    а~\( B \) имеет единицы в~точности на тех \( M \) местах,
    где в~\( A \) стоят нули.
Это граф, полученный из полного графа
    удалением некоторых \( M \) направленных рёбер.

Спектр \( \sigma\left( \mathcal{J}_N \right) \)
    матрицы \( \mathcal{J}_N \) легко считается:
    \( \mathcal{J}_N^2 = N \mathcal{J}_N, \) т.е.
    \( \lambda(\lambda - N) \)~--- аннулирующий и, что легко проверить,
    минимальный многочлен матрицы \( \mathcal{J}_N \), а~значит
    \( \sigma\left( \mathcal{J}_N \right) = \left\{ 0,N \right\}. \)
Ядро и образ этой матрицы ортогональны.
Ненулевому собственному значению
    отвечает собственный вектор
    \[
        h_N = \frac{1}{\sqrt{N}}(1,\ldots, 1)\in\mathbb{R}^N,
        \]
    а ортонормальный базис ядра \( \operatorname{Ker}(J_N) \)
    составляют векторы
    \[
        h_k = \frac{1}{k(k+1)}
        \big(\underbrace{1,\ldots, 1}_{k\ \text{раз}}, -k, 0, \ldots, 0\big)
        \in\mathbb{R}^N,
        \ k=\overline{1,N{-1}}.
        \]

Так, матрица \( J_N-B \) подобна (блочной) матрице
    \[
        \mathcal{A} - \mathcal{B}, \]
    \[
        \mathcal{A} = \left(\begin{array}{c|c}
        N & \mathbf{0} \\ \hline
        \mathbf{0} & \mathbf{0}
        \end{array}\right),
        \]
    \[
        \mathcal{B} = U^{-1}B U,
        \]
    где \( U=\operatorname{columns}(h_N, h_1, \ldots, h_{N-1}) \)
    --- матрица, составленная из собственных векторов-столбцов матрицы \( J_N \).

Построим допустимую тройку для матрицы \( \mathcal{A} \).
Проектор \( J \) естественно задать формулой
    \[
        JX =
        \left(\begin{array}{c|c}
        x_{11} & \mathbf{0} \\ \hline
        \mathbf{0} & X_{22}
        \end{array}\right)
        \]
где
    \[
        X =
        \left(\begin{array}{c|c}
        x_{11} & X_{12} \\ \hline
        X_{21} & X_{22}
        \end{array}\right)\in\mathbb{K}^{n{\times}n}
        \]
        --- записанная в блочном виде матрица,
        \( x_{11}\in\mathbb{R}^{1\times 1} \) --- число.
При этом матрица \( \mathcal{A} - JX \),
    \( X\in\mathbb{R}^{n{\times}n} \),
    является блочно-диагональной,
    и её спектр есть объединение спектров её диагональных блоков:
    \[
        \sigma(\mathcal{A} - JX)
        = \{ N - x_{11} \} \cup \sigma(-X_{22}).
        \]

Теперь найдём соответствующий \( J \)
    трансформатор \( \Gamma \).
Пусть \( \Gamma X = \begin{pmatrix}
    \Gamma_{11}(X) & \Gamma_{12}(X) \\
    \Gamma_{21}(X) & \Gamma_{22}(X)
\end{pmatrix} \).
Тогда
\[
    \mathcal{A} \Gamma X - (\Gamma X)\mathcal{A} =
    N
    \begin{pmatrix}
        0 & \Gamma_{12}(X) \\
        -\Gamma_{21}(X) & 0
     \end{pmatrix},
     \]
Из этого и из \( J\Gamma X = 0 \) находим
    \[
        \Gamma X = \frac{1}{N} \begin{pmatrix} 0 & X_{12} \\ -X_{21} & 0 \end{pmatrix}.
        \]
При этом
    \(
        \|\Gamma\|_{\mathrm{op}} \leq \frac1N.
        \)

Рассматривая теперь в пространстве \( \mathbb{R}^{n{\times}n} \)
    норму Фробениуса \( \|\cdot\|_{\mathrm{F}} \)
    и учитывая неравенство \( \|X\|_{\mathrm{op}} \leq \|X\|_{\mathrm{F}} \),
    получаем \( \|\mathcal{B}\|_{\mathrm{op}} =
        \|B\|_{\mathrm{op}} \leq \|B\|_{\mathrm{F}} = \sqrt{M} \),
    где \( M \) --- число единиц в матрице \( B \).

Непосредственно следует
\begin{ksvthm}\label{kozlukovsv:thm:almost-all-ones}
    Пусть \( M < \frac{1}{16}N^2 \),
        тогда спектр матрицы \( A = J_N{-}B \) можно представить в~виде
        объединения \( \sigma\left(A\right) = \sigma_1 \cup \sigma_2 \)
        непересекающихся
        одноэлементного множества \( {\sigma_1 = \left\{ N{-}x_{11}^o \right\}} \)
        и~множества \( \sigma_2=\sigma(-X_{22}^o) \), удовлетворяющих условиям:
    \[
        |x_{11}^o| \leq 4\sqrt{M},
        \]
    \[
        \sup_{\lambda\in\sigma_2} |\lambda| \leq 4\sqrt{M}.
        \]
    Собственное значение \( N - x_{11}^o \) совпадает с спектральным радиусом
        \( \operatorname{spr}(A) \),
        и ему соответствует собственный вектор
        \[
            \hat{h}_N =
            U(E+\Gamma X)\begin{pmatrix}1\\0\\ {\vdots}\\ 0\end{pmatrix} =
            h_N -
            \frac1N (
                {X_{21}^o}^{(1)} h_1 +
                \cdots +
                {X_{21}^o}^{(N-1)} h_{N-1}),
            \]
        где \( {X_{21}^o}^{(1)}, \ldots, {X_{21}^o}^{(N-1)} \)
        --- координаты вектора \( X_{21}^o\in\mathbb{R}^{(N{-}1){\times}1} \).
    При этом верна оценка
    \[
        \left\|\hat{h}_N - h_N \right\|_2 \leq 4\frac{\sqrt{M}}{N}.
        \]
\end{ksvthm}
\subsection*{Пример 2. Матрица, составленная из однородных блоков}
Пусть теперь \( A\in \mathbb{K}^{m\times m} \)
Будем рассматривать матрицу
\[
    \mathbb{A} =
    \begin{pmatrix}
        A      & \cdots & A \\
        \vdots & \ddots & \vdots \\
        A      & \cdots & A
    \end{pmatrix}
    \in \mathbb{K}^{mn{\times}mn},
    \]
    составленную из \( n\times n \) блоков
    и возмущённую матрицу
\[
    \mathbb{A} - \mathbb{B}.
    \]

\begin{ksvlem}
    Пусть \( A \) --- обратимая самосопряжённая матрица.
    В этом случае она имеет \( m \) ортонормальных
    собственных векторов \( h_1, \ldots, h_m \)
    (\( \|h_k\|_2=1,\ k{=}\overline{1,m} \)),
    которым отвечают собственные значения\( \lambda_1, \ldots, \lambda_m \).
    \( \lambda_1, \ldots, \lambda_m \)
    --- её собственные значения,
    с соответствующими им собственными векторами
    \( h_1, \ldots, h_m \),

    Тогда спектр матрицы \( \mathbb{A} \) есть
    \[
        \sigma(\mathbb{A}) = \{0\} \cup n\sigma(A)
        = \{ 0 \} \cup \{ n\lambda;\ \lambda\in\sigma(A) \}.
        \]
    Ненулевым собственным значениям \( n\lambda_1, \ldots, n\lambda_m \)
    матрицы \( \mathbb{A} \)
    соответствуют блочные собственные векторы
    \[
        f_j = \frac{1}{\sqrt{n}}(h_k, \ldots, h_k)\in\mathbb{K}^{mn},
        \ j{=}\overline{1,m},
       \]
    а ортобазис ядра образуют ортогональные образу
    векторы
    \[
        f_{j,k} =
        \frac{1}{\sqrt{k(k+1)}}
        (\underbrace{e_j,\ldots,e_j}_{k\ \text{раз}}, -ke_j, 0, \ldots, 0)\in\mathbb{K}^{mn},
        \ k{=}\overline{1,n-1},
        \ j{=}\overline{1,m}.
        \]
    Другими словами, матрица
    \[
        \mathbb{A} =
        \begin{pmatrix}
            A      & \cdots & A \\
            \vdots & \ddots & \vdots \\
            A      & \cdots & A
        \end{pmatrix}
        \in\mathbb{K}^{mn{\times}mn}
        \]
    подобна блочно-диагональной матрице
    \[
        \mathcal{A} =
        \left(\begin{array}{c|c}
            \operatorname{diag}(n\lambda_1,\ldots,n\lambda_m) & 0 \\ \hline
            0 & \mathbf{0}
        \end{array}\right),
        \]
    с ортогональной матрицей преобразования подобия
    \[
        U = \operatorname{columns}
        \left(f_1, \ldots, f_m, f_{1,1}, \ldots, f_{1,n{-1}}, \ldots, f_{m,n{-}1}\right).
        \]
\end{ksvlem}

Перейдём к изучению возмущённой матрицы.
Матрицы будем рассматривать в блочном виде:
\[
    X =
        \left(\begin{array}{c|c}
            \begin{matrix}
                x_{11} & \cdots & x_{1m} \\
                \vdots & \ddots & \vdots \\
                x_{m1} & \cdots & x_{mm}
            \end{matrix} &
            \begin{matrix}
                x_{1,m+1} \\
                \vdots \\
                x_{m,m+1}
            \end{matrix} \\ \hline
            \begin{matrix}
                x_{m+1,1} &
                \cdots &
                x_{m+1,m}
            \end{matrix} &
            X_{m+1,m+1}
        \end{array}\right),
    \]
где
\( X_{ij}      {=} x_{ij}      {\in} \mathbb{K} \),
\( X_{m{+}1,j} {=} x_{m{+}1,j} {\in} \mathbb{K} \),
\( X_{i,m{+}1} {=} x_{i,m{+}1} {\in} \mathbb{K} \)
для \( 1\leq i,j \leq m \),
а в нижнем правом углу расположен блок
\( X_{m{+}1,m{+}1} {\in} \mathbb{K}^{m(n{-}1){\times}m(n-1)} \).
Как и прежде, в качестве \( J \)
возьмём оператор блочной диагонализации:
\[
    X =
        \left(\begin{array}{c|c}
            \begin{matrix}
                x_{11} &  & 0 \\
                 & \ddots &  \\
                0 &  & x_{mm}
            \end{matrix} &
            \begin{matrix}
                0 \\
                \vdots \\
                0
            \end{matrix} \\ \hline
            \begin{matrix}
                0 & \cdots & 0
            \end{matrix} &
            X_{m+1,m+1}
        \end{array}\right),
    \]
\begin{ksvlem}
    Пусть все числа \( \lambda_1, \ldots, \lambda_m \) различны.
    Тогда тройка \( (\mathbb{K}^{m{\times}n}, J, \Gamma) \)
    с трансформатором \( \Gamma \), действующим по формуле
    \[
        \Gamma X =
        \left(\begin{array}{c|c}
            \begin{matrix}
                0               & \gamma_{12}(X) & \cdots & \gamma_{1m}(X) \\
                \gamma_{21}(X)  & 0              & \cdots & \gamma_{2m}(X) \\
                \vdots          & \vdots         & \ddots & \vdots & \ \\
                \gamma_{m1}(X)  & \gamma_{m2}(X) & \cdots & 0
            \end{matrix} &
            \begin{matrix}
                \gamma_{1,m+1}(X) \\
                \gamma_{2,m+1}(X) \\
                \vdots \\
                \gamma_{m,m+1}(X)
            \end{matrix} \\ \hline
            \begin{matrix}
                \gamma_{m{+}1,1}(X) &
                \gamma_{m{+}1,2}(X) &
                \cdots &
                \gamma_{m{+}1,m}(X)
            \end{matrix} &
            \mathbf{0}
        \end{array}\right),
        \]
    \[
        \gamma_{ij}(X) =
        \left\{\begin{aligned}
            & \frac1N \frac{1}{\lambda_i - \lambda_j} x_{ij},
              \ 1\leq i{\neq}j \leq m{+}1, \\
            & 0,
              \ 1\leq i{=}j \leq m{+}1,
        \end{aligned}\right.
        \]
    \[
        \lambda_{m+1} \coloneqq 0,
        \]
    является допустимой.
    При этом
    \[
        \|\Gamma\|_{\mathrm{op}} =
        \frac1N
        \frac{1}{\min_{1\leq i{\neq}j \leq m{+}1}|\lambda_i - \lambda_j|} =
        \]
    \[
        = \frac1N
         \max\left\{
         \frac{1}{
             \min\limits_{1\leq i{\neq}j \leq m }{|\lambda_i - \lambda_j|}},
         \frac{1}{
             \min\limits_{1\leq j \leq m}{|\lambda_j|}}
         \right\}
        \]
\end{ksvlem}

\begin{ksvthm}
    Пусть
    \[
        \left\| \mathbb{B} \right\|_{\mathrm{op}}
        \leq
        \frac{N}{4}
         \min\left\{
             \min\limits_{1\leq i{\neq}j \leq m }{|\lambda_i - \lambda_j|},
             \min\limits_{1\leq j \leq m}{|\lambda_j|}
         \right\}.
        \]
    Тогда исследуемая возмущённая матрица
    \[
        \mathbb{A} - \mathbb{B}
        \]
    имеет спектр
    \[
        \sigma\left(\mathbb{A}\right) =
        \left\{
            n\lambda_1 - x_{11}^o, \ldots, n\lambda_m - x_{mm}^o, 0
        \right\}
        \]
    и собственные векторы
    \[
        \hat{f}_j = f_j + \xi_j,
        ,\ j{=}\overline{1,m}
        \]
    \[
        \hat{f}_{j,k} = f_{j,k} + \xi_{j,k},
        \ j{=}\overline{1,m},
        \ k{=}\overline{1,N{-}1},
        \]
    где
    \[
        x_{jj}{\in}\mathbb{C},
        \ \xi_j, \xi_{j,k}{\in}\mathbb{K}^{mn},
        \]
    \[
        |x_{jj}|,\ \|\xi_j\|_2, \|\xi_{j,k}\|_2 \leq
        4 \|\mathbb{B}\|_{\mathrm{op}}
          \frac1N
          \max\left\{
          \frac{1}{
              \min\limits_{1\leq i{\neq}j \leq m }{|\lambda_i - \lambda_j|}},
          \frac{1}{
              \min\limits_{1\leq j \leq m}{|\lambda_j|}}
          \right\},
        \]
    \[
        j{=}\overline{1,m},
        \ k{=}\overline{1,N{-}1}.
        \]
\end{ksvthm}
\subsection*{Направление дальнейших исследований: прямое произведение (произведение Кронекера)}
Прямое произведение (произведение Кронекера)
    \( X\otimes Y \)
    двух матриц
    \( X{=}(x_{ij}){\in}\mathbb{K}^{m_1{\times}n_1},\ \)
    \( Y{\in}\mathbb{K}^{m_2{\times}n_2} \).
    определено как блочная матрица
\[
    X\otimes Y =
    \begin{pmatrix}
        x_{11} Y & \cdots & x_{1n_1} Y \\
        \vdots   & \ddots & \vdots \\
        x_{m_11} Y & \cdots & x_{m_1n_1} Y \\
    \end{pmatrix}\in\mathbb{K}^{m_1m_1\times n_1n_2}.
    \]

Пусть \( A{=}(a_{ij}){\in}\mathbb{K}^{n\times n} \)
    и \( B\in\mathbb{K}^{m\times m} \).
Для расширения класса допустимых матриц,
    можно исследовать спектральные свойства прямого произведения
    \[
        A\otimes B =
        \begin{pmatrix}
            a_{11} B & \cdots & a_{1n} B \\
            \vdots   & \ddots & \vdots \\
            a_{n1} B & \cdots & a_{nn} B \\
        \end{pmatrix}\in\mathbb{K}^{mn\times mn}
        \]
    и соответствующей возмущённой матрицы
    \begin{equation}\label{ksv:disturbkron}
        A\otimes B - \mathbb{F},
    \end{equation}
    имея в виду следующий факт:

\begin{ksvlem}
    Пусть \( h\in\mathbb{K}^n \) --- собственный вектор матрицы \( A \),
    отвечающий собственному значению \( \lambda \),
    а \( f\in\mathbb{K}^m \) --- собственный вектор,
    соответствующий собственному значению \( \mu \):
    \[
        A h = \lambda h,
        \]
    \[
        B f = \mu f.
        \]
    Тогда \( h\otimes f \in \mathbb{K}^{mn} \)
    --- собственный вектор матрицы \( A\otimes B \),
    отвечающий собственному значению \( \lambda\mu \):
    \[
        (A\otimes B)(h\otimes f) = \lambda\mu\ h\otimes f.
        \]
\end{ksvlem}
\begin{proof}
    Для доказательства достаточно расписать выражение по координатам.
\end{proof}

Заметим, что рассмотренная в предыдущем разделе
    блочная матрица
    представляет собой частный случай \eqref{ksv:disturbkron}:
\[
    \begin{pmatrix}
        B      & \cdots & B \\
        \vdots & \ddots & \vdots \\
        B      & \cdots & B
    \end{pmatrix} =
    J_n \otimes B,
        \]
    где \( J_n \) --- матрица единиц,
    исследованная в первом примере.
В соответствии с леммой спектр
    матрицы \( J_n \otimes B \)
    есть \( \{ 0 \} \cup n\sigma(B) \).
Заметим, что матрица \( J_n \)
    --- сингулярная, простой структуры с кратными собственными значениями:
    её собственному значению \( 0 \)
    соответствует \( n{-}1 \) линейно-независимых
    собственных векторов.
Как мы видели в предыдущем пункте,
    из-за требования попарной различности собственных значений,
    это привело к появлению  в матрице \( J X \) соответствующего ядру
    блока размера \( m(n{-}1)\times m(n{-}1) \).

Таким образом, в будущем можно обобщить результаты
    на случай прямого произведения матриц
    (тензорного произведения операторов~\cite{abraham2012manifolds}),
    а~также избавиться от ограничений на обратимость и самосопряжённость.

\smallskip\centerline{\bf Литература}
\begingroup
    \renewcommand{\section}[2]{}%
% Bibliography;
% In a release
% the %.bbl file must be supplied.
% It'll be `\input`-ed directly,
% thus bibtex dependency is avoided.
    \IfFileExists{current-problems-2017.bbl}{
\begin{thebibliography}{10}

\bibitem{abraham2012manifolds}
Ralph Abraham, Jerrold~E Marsden, and Tudor Ratiu.
\newblock {\em Manifolds, tensor analysis, and applications}, volume~75.
\newblock Springer Science \& Business Media, 2012.

\bibitem{bonacich1972factoring}
Phillip Bonacich.
\newblock Factoring and weighting approaches to status scores and clique
  identification.
\newblock {\em Journal of Mathematical Sociology}, 2(1):113--120, 1972.

\bibitem{chakrabarti2008epidemic}
Deepayan Chakrabarti, Yang Wang, Chenxi Wang, Jurij Leskovec, and Christos
  Faloutsos.
\newblock Epidemic thresholds in real networks.
\newblock {\em ACM Transactions on Information and System Security (TISSEC)},
  10(4):1, 2008.

\bibitem{cvetkovic1980spectra}
D~Cvetkovi{\'c}, M~Doob, and H~Sachs.
\newblock Spectra of graph-theory and applications, 1980.

\bibitem{meyer1994sensitivity}
Carl~D Meyer.
\newblock Sensitivity of the stationary distribution of a markov chain.
\newblock {\em SIAM Journal on Matrix Analysis and Applications},
  15(3):715--728, 1994.

\bibitem{ilprints422}
Lawrence Page, Sergey Brin, Rajeev Motwani, and Terry Winograd.
\newblock The pagerank citation ranking: Bringing order to the web.
\newblock Technical Report 1999-66, Stanford InfoLab, November 1999.
\newblock Previous number = SIDL-WP-1999-0120.

\bibitem{schwartz2006fluctuation}
Ira~B Schwartz and Lora Billings.
\newblock Fluctuation induced almost invariant sets.
\newblock Technical report, NAVAL RESEARCH LAB WASHINGTON DC PLASMA PHYSICS
  DIV, 2006.

\bibitem{wang2003epidemic}
Yang Wang, Deepayan Chakrabarti, Chenxi Wang, and Christos Faloutsos.
\newblock Epidemic spreading in real networks: An eigenvalue viewpoint.
\newblock In {\em Reliable Distributed Systems, 2003. Proceedings. 22nd
  International Symposium on}, pages 25--34. IEEE, 2003.

\bibitem{baskakov1983}
А.Г. Баскаков.
\newblock Методы абстрактного гармонического
  анализа в теории возмущений линейных
  операторов.
\newblock {\em Sibirskij matematiceskij zurnal}, 24(1):21--39, 1983.

\bibitem{baskakov-harmonic}
Анатолий~Григорьевич Баскаков.
\newblock Гармонический анализ линейных
  операторов.
\newblock 1987.

\end{thebibliography}
    }{
            \bibliography{current-problems-2017}{}
            \bibliographystyle{unsrt}
    }
\endgroup
\end{document}
