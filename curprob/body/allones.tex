Рассмотрим граф, заданный матрицей смежностей
\[ A = \mathcal{J}_N - B =
   \begin{pmatrix}1 & \cdots & 1\\ \vdots & \ddots & \vdots \\ 1 & \cdots & 1\end{pmatrix} - B,
   \]
    где \( \mathcal{J}_N \)~--- матрица, составленная из \( N{\times}N \) единиц,
    а~\( B \) имеет единицы в~точности на тех \( M \) местах,
    где в~\( A \) стоят нули.
Это граф, полученный из полного графа
    удалением некоторых \( M \) направленных рёбер.

Спектр \( \sigma\left( \mathcal{J}_N \right) \)
    матрицы \( \mathcal{J}_N \) легко считается:
    \( \mathcal{J}_N^2 = N \mathcal{J}_N, \) т.е.
    \( \lambda(\lambda - N) \)~--- аннулирующий и, что легко проверить,
    минимальный многочлен матрицы \( \mathcal{J}_N \), а~значит
    \( \sigma\left( \mathcal{J}_N \right) = \left\{ 0,N \right\}. \)
Ядро и образ этой матрицы ортогональны.
Ненулевому собственному значению
    отвечает собственный вектор
    \[
        h_N = \frac{1}{\sqrt{N}}(1,\ldots, 1)\in\mathbb{R}^N,
        \]
    а ортонормальный базис ядра \( \operatorname{Ker}(J_N) \)
    составляют векторы
    \[
        h_k = \frac{1}{\sqrt{k(k+1)}}
        \big(\underbrace{1,\ldots, 1}_{k\ \text{раз}}, -k, 0, \ldots, 0\big)
        \in\mathbb{R}^N,
        \ k=\overline{1,N{-1}}.
        \]

Так, матрица \( J_N-B \) подобна (блочной) матрице
    \[
        \mathcal{A} - \mathcal{B}, \]
    \[
        \mathcal{A} = \left(\begin{array}{c|c}
        N & \mathbf{0} \\ \hline
        \mathbf{0} & \mathbf{0}
        \end{array}\right),
        \]
    \[
        \mathcal{B} = U^{-1}B U,
        \]
    где \( U=\operatorname{columns}(h_N, h_1, \ldots, h_{N-1}) \)
    --- матрица, составленная из собственных векторов-столбцов матрицы \( J_N \).

Построим допустимую тройку для матрицы \( \mathcal{A} \).
Проектор \( J \) естественно задать формулой
    \[
        JX =
        \left(\begin{array}{c|c}
        x_{11} & \mathbf{0} \\ \hline
        \mathbf{0} & X_{22}
        \end{array}\right)
        \]
где
    \[
        X =
        \left(\begin{array}{c|c}
        x_{11} & X_{12} \\ \hline
        X_{21} & X_{22}
        \end{array}\right)\in\mathbb{K}^{n{\times}n}
        \]
        --- записанная в блочном виде матрица,
        \( x_{11}\in\mathbb{R}^{1\times 1} \) --- число.
При этом матрица \( \mathcal{A} - JX \),
    \( X\in\mathbb{R}^{n{\times}n} \),
    является блочно-диагональной,
    и её спектр есть объединение спектров её диагональных блоков:
    \[
        \sigma(\mathcal{A} - JX)
        = \{ N - x_{11} \} \cup \sigma(-X_{22}).
        \]

Теперь найдём соответствующий \( J \)
    трансформатор \( \Gamma \).
Пусть \( \Gamma X = \begin{pmatrix}
    \Gamma_{11}(X) & \Gamma_{12}(X) \\
    \Gamma_{21}(X) & \Gamma_{22}(X)
\end{pmatrix} \).
Тогда
\[
    \mathcal{A} \Gamma X - (\Gamma X)\mathcal{A} = 
    N
    \begin{pmatrix}
        0 & \Gamma_{12}(X) \\
        -\Gamma_{21}(X) & 0
     \end{pmatrix},
     \]
Из этого и из \( J\Gamma X = 0 \) находим
    \[
        \Gamma X = \frac{1}{N} \begin{pmatrix} 0 & X_{12} \\ -X_{21} & 0 \end{pmatrix}.
        \]
При этом
    \(
        \|\Gamma\|_{\mathrm{op}} \leq \frac1N.
        \)

Рассматривая теперь в пространстве \( \mathbb{R}^{n{\times}n} \)
    норму Фробениуса \( \|\cdot\|_{\mathrm{F}} \)
    и учитывая неравенство \( \|X\|_{\mathrm{op}} \leq \|X\|_{\mathrm{F}} \),
    получаем \( \|\mathcal{B}\|_{\mathrm{op}} =
        \|B\|_{\mathrm{op}} \leq \|B\|_{\mathrm{F}} = \sqrt{M} \),
    где \( M \) --- число единиц в матрице \( B \).

Непосредственно следует
\begin{ksvthm}\label{kozlukovsv:thm:almost-all-ones}
    Пусть \( M < \frac{1}{16}N^2 \),
        тогда спектр матрицы \( A = J_N{-}B \) можно представить в~виде
        объединения \( \sigma\left(A\right) = \sigma_1 \cup \sigma_2 \)
        непересекающихся
        одноэлементного множества \( {\sigma_1 = \left\{ N{-}x_{11}^o \right\}} \)
        и~множества \( \sigma_2=\sigma(-X_{22}^o) \), удовлетворяющих условиям:
    \[
        |x_{11}^o| \leq 4\sqrt{M},
        \]
    \[
        \sup_{\lambda\in\sigma_2} |\lambda| \leq 4\sqrt{M}.
        \]
    Собственное значение \( N - x_{11}^o \) совпадает с спектральным радиусом
        \( \operatorname{spr}(A) \),
        и ему соответствует собственный вектор
        \[
            \hat{h}_N =
            U(E+\Gamma X^o)\begin{pmatrix}1\\0\\ {\vdots}\\ 0\end{pmatrix} =
            h_N -
            \frac1N (
                {X_{21}^o}^{(1)} h_1 + 
                \cdots +
                {X_{21}^o}^{(N-1)} h_{N-1}),
            \]
        где \( {X_{21}^o}^{(1)}, \ldots, {X_{21}^o}^{(N-1)} \)
        --- координаты вектора \( X_{21}^o\in\mathbb{R}^{(N{-}1){\times}1} \).
    При этом верна оценка
    \[
        \left\|\hat{h}_N - h_N \right\|_2 \leq 4\frac{\sqrt{M}}{N}.
        \]
\end{ksvthm}
